% REMEMBER: You must not plagiarise anything in your report. Be extremely careful.

\documentclass{l4proj}
%
% put any additional packages here
%
\usepackage{todonotes}
\usepackage{booktabs}
\usepackage{diagbox}
\usepackage[table,xcdraw]{xcolor}

\begin{document}

%==============================================================================
%% METADATA
\title{Deep learning for analysing immune cell interactions}
\author{Leonore Papaloizos}
\date{\today}

\maketitle

%==============================================================================
%% ABSTRACT
\begin{abstract}
    The protective responses of our immune system are initiated by interactions between immune cells. These interactions can be inhibited or enhanced by the application of drugs. This project investigated using deep learning models to analyse microscope images of immune cells to study their interactions under different experimental conditions. Deep learning has scarcely been applied in the field of immunology. We implemented and evaluated a convolutional autoencoder and an autoencoder-based regression model to qualitatively and quantitatively analyse the interaction between T cells and dendritic cells in microscope images. We found that the autoencoder helped speed up the process of data visualisation and that the regression model successfully predicted measures of interaction in unseen images of immune cells. With carefully selected and pre-processed datasets, deep learning can be a useful technique for immunology researchers to analyse immune cell interaction.
\end{abstract}

%==============================================================================

% EDUCATION REUSE CONSENT FORM
% If you consent to your project being shown to future students for educational purposes
% then insert your name and the date below to  sign the education use form that appears in the front of the document. 
% You must explicitly give consent if you wish to do so.
% If you sign, your project may be included in the Hall of Fame if it scores particularly highly.
%
% Please note that you are under no obligation to sign 
% this declaration, but doing so would help future students.
%
\def\consentname {Leonore Papaloizos} % your full name
\def\consentdate {\today} % the date you agree
%
\educationalconsent

%==============================================================================

\renewcommand{\abstractname}{Acknowledgements}
\begin{abstract}
I would like to thank Dr Carol Webster and Dr Hannah Scales for their continuous guidance and advice throughout this project. Thank you to all the friends and family that supported and encouraged me in the past four years.
\end{abstract}

%==============================================================================
\tableofcontents

%==============================================================================
%% Notes on formatting
%==============================================================================
% The first page, abstract and table of contents are numbered using Roman numerals and are not
% included in the page count. 
%
% From now on pages are numbered
% using Arabic numerals. Therefore, immediately after the first call to \chapter we need the call
% \pagenumbering{arabic} and this should be called once only in the document. 
%
% Do not alter the bibliography style.
%
% The first Chapter should then be on page 1. You are allowed 40 pages for a 40 credit project and 30 pages for a 
% 20 credit report. This includes everything numbered in Arabic numerals (excluding front matter) up
% to but excluding the appendices and bibliography.
%
% You must not alter text size (it is currently 10pt) or alter margins or spacing.
%
%
%==================================================================================================================================
%
% IMPORTANT
% The chapter headings here are **suggestions**. You don't have to follow this model if
% it doesn't fit your project. Every project should have an introduction and conclusion,
% however. 
%
%==================================================================================================================================

\chapter{Introduction}

\section{Motivation}

Picture your immune system as a speed date. Your immune cells go on quick dates with other immune cells, who tell them about their life. The immune cells might be more or less under the influence of substances. The success of the discussion between the two immune cells during their speed date determines how your immune system is going to react as a whole. The consumption of substances might help make the reaction more positive, or might make a situation worse. Your immune system might be pleased with the date, and react positively, or an immune cell might get offended by their date, and trigger a negative response. % These dates could happen very fast with immune cells moving on to their next date, but could also be prolonged into longer relationships.

%The two partners we are studying are T cells and dendritic cells. 
%long-term, non monogamous

% biological response
Formally put, the initiation of an immune response in our immune system depends on the interaction strength between different types of immune cells. Certain types of immune cells relay information about their environment to other types immune cells, who can then trigger an appropriate response depending on the what they have learned from the other cell about the environment. The environment might contain substances like foreign, dangerous bodies. These interactions between immune cells can be enhanced or inhibited by the application of drugs. Studying the reactions of immune cells under the influence different types of drugs is key in the development of drugs for diseases such as viral infections or auto-immune diseases. 

\section{General problem and our idea}

One way of studying the interaction of immune cells is through microscopic images obtained in artificial settings. We can place immune cells in a dish and study them under a microscope under different experimental conditions, which could be different types of drugs being injected into the dish. Microscopic images of these cells can then be systematically captured.

We are however not interested in how these images are captured, but how they can be analysed. Microscopic images are often analysed with proprietary software which is costly to maintain and whose inner workings are hard to understand. On the other hand, applying deep learning techniques to biomedical data is becoming increasingly popular and more accessible, while obtaining promising results. The problem we are looking at is applying deep learning to the analysis of images of immune cells. We want to explore how deep learning could be used to systematically explore immune cell interaction from microscopic images. The aim is to assess whether using deep learning in the field of immunology can provide any useful information on interaction levels between immune cells under different experimental conditions. 
\pagenumbering{arabic} % reset page numbering. Don't remove this!

\chapter{Background}

\section{Immunology concepts}

\subsection{Our immune system} \label{bg:immunesystem}

Our immune system consists of organs, cells and groups of cells working in collaboration to defend us from other organisms that could pose a danger to our health. Such outside forces could be harmful viruses, bacteria or parasites for example. The human body is a haven for these to thrive in, at our detriment. Our immune system protects us by attacking these foreign bodies – defined as antigens – when they are detected and filed as dangerous. The key in this exchange is for our immune system to recognise which biological entities are ours, and which are alien, potentially dangerous elements (\cite{http://www.imgt.org/IMGTeducation/Tutorials/ImmuneSystem/UK/the_immune_system.pdf}).

\begin{figure}[h]
    \centering
    \begin{subfigure}[h!]{0.3\textwidth}
        \includegraphics[width=\textwidth]{dissertation/figures/example_DCs.png}
    \end{subfigure}
    \begin{subfigure}[h!]{0.3\textwidth}
        \includegraphics[width=\textwidth]{dissertation/figures/example_Tcells.png}
    \end{subfigure}
    \caption{Microscopic images of dendritic cells (left) and T cells (right) in an assay. Scale: , 200x zoom.}
\end{figure}

\begin{figure}[h]
    \centering
    \begin{subfigure}[h!]{0.3\textwidth}
        \includegraphics[width=\textwidth]{dissertation/figures/model_DC.png}
    \end{subfigure}
    \begin{subfigure}[h!]{0.3\textwidth}
        \includegraphics[width=\textwidth]{dissertation/figures/model_Tcell.png}
    \end{subfigure}
    \caption{Model of a dendritic cell (left) and a T cell (right). Adapted from \cite{https://www.immunology.org/public-information/bitesized-immunology/systems-and-processes/t-cell-activation}}
    \label{eval:graphs}
\end{figure}

% What is the scale of the microscopic images?

The actors of our immune systems are thus the key defenders of our bodies. The actors we are interested in for the purpose of this research are T lymphocytes – ``T cells" – and dendritic cells – DCs. Dendritic cells are ``sentinels" and initiate our immune system's responses by sensing and integrating information from their environment and sending it over to T-cells. T-cells are ``master controllers" and trigger the appropriate immune response, if any, from the information they have received, notably from dendritic cells, often in the form of chemical signals or intercellular interactions (\cite{https://www.immunology.org/public-information/bitesized-immunology/cells/dendritic-cells, https://www.youtube.com/watch?v=hRvyCYyab68}). [is this sufficient? don't want to give too much information that would lose the reader]

%They interact with T cells, interactions which might be transient if nothing is to be triggered, or more communicative (\cite{https://www.youtube.com/watch?v=hRvyCYyab68}).

%The idea is that we can stimulate this interaction between dendritic cells and T cells through different drug components, both for inhibition or increase of level of interaction.

\subsection{Effects of interaction between immune cells} \label{bg:interaction}

Antigens, a generic term for all structures recognised as threats by our immune system, can be fought by antibodies, which are defensive proteins produced by our immune system. More specifically, antibodies are produced by some immune cells in a process which starts in T cells, and in some cases is activated by T cells seeing antigens on the surface of dendritic cells (\cite{https://elifesciences.org/articles/06994}). Hence, the interaction between dendritic cells and T cells is critical in the decision for our immune system to produce agents to defend our body.

The purpose of this dissertation is to evaluate how much interaction is observed between immune cells. There is existing work in the field of immunology looking into the effects of these changes in interaction. Benson et al. show how the generation of antibodies might be impacted by T cell and dendritic cell interaction. They studied how dendritic cells and T-cells interacted in the mouse immune system, both in terms of whether or not interaction was witnessed, or of duration of interaction. This interaction was studied under different conditions, with different drug compounds being used to attempt to drive interaction or to inhibit it. They found that under conditions where compounds were blocking interaction between T-cells and DCs, less antibodies were generated, meaning that the mice were not defending themselves as much. Hence, the study of the impact of compounds on the interactions between immune cells can tell us how our immune system will then operate.

\subsection{Implications}

Concepts and research described in Sections \ref{bg:immunesystem} and \ref{bg:interaction} show that changes in interactions between immune cells control the way in which our immune system protects itself. Hence, analysing the interaction between immune cells under different experimental conditions bears a particular interest in the field of immunology for studying immune responses. We want to analyse this interaction with the help of deep learning techniques.

% title: qualifying interaction
\section{Concepts of interest in Deep Learning}

The following sections collate selected research that show how deep learning techniques could be applied in our context of study.

\subsection{Convolutional operations for image feature extraction}

%A number of approaches already use deep neural networks for classification from textual genome sequences, particularly for cancer type detection (\cite{https://www.biorxiv.org/content/10.1101/612762v1}, \cite{https://www.nature.com/articles/s41598-019-53989-3}).
Convolutional operations in neural networks were first introduced by Fukushima at the start of the 1980s for pattern recognition. They were later popularised by LeCun as a method for object recognition, once back-propagation was put to use as a learning procedure for networks. LeCun applied his convolutional neural network to digit recognition and classification. Since then, convolutional operations in neural networks have proven to be successful to extract features from more complicated images. In a recent medical example, Shen et al. trained a convolutional neural network structure to detect breast cancer from mammography screenings and which showed competitive results compared to commercial systems (\cite{https://www.nature.com/articles/s41598-019-48995-4}.
[how much research to cite in this context? there is a lot available]

\subsection{Autoencoders for dimensionality reduction}

An autoencoder is a type of neural network trained to map its input to its output from a compressed representation of the input, as shown in Figure \ref{fig:autoencoder}. The compressed representation of the input obtained from the bottleneck layer is a \textit{coded} representation of the input, while the final output of the network is the \textit{decoded} version of the input. Autoencoders are not trained to learn a perfect copy of the input data, but a smaller, compressed copy with features which the neural network learns to be most important to be able to gain and overall understanding of the input. Autoencoders were first introduced in the 1980s (LeCun, DE Rumelhart) and are traditionally used for dimensionality reduction and feature extraction (\cite{http://www.deeplearningbook.org/contents/autoencoders.html}).

\begin{figure}[h!]
    \centering
    \includegraphics[width=0.45\textwidth]{dissertation/figures/autoencoder_schema.png}
    \caption{Autoencoder representation. To be adapted in Photoshop?. Current source: Wikipedia}
    \label{fig:autoencoder}
\end{figure}

Zamparo and Zhang, 2015 (\cite{https://arxiv.org/pdf/1501.01348.pdf}) show that autoencoders can be successfully applied for dimensionality reduction in the context of biomedical data. Their autoencoder approach, applied to the unsupervised clustering of cell phenotypes, outperformed other dimensionality reduction techniques such as Principal Component Analysis (PCA). However, the context of this approach was not on imaging data.

% http://arxiv.org/abs/1809.00027
Nonetheless, autoencoders have been successfully used for reducing the dimensionality of large imaging data by using convolution operations in their structure. Saenz et al. successfully used convolutional autoencoders for feature extraction from climate imaging data. %More famously, convolutional autoencoders have been shown to be efficient in improving the clustering of the MNIST dataset, when visualised with high-visualising techniques like t-sne and UMAP. t-sne and UMAP are of interest to us as they allow to project high-dimensional data onto 2 dimensions (or three in the case UMAP). Using them allows us to assess whether or not our data has some underlying structure – e.g. whether or not we can cluster cells according to different experimental conditions.

\subsection{Deep regression models}

Neural networks are only limited by logic and can be constructed for regression tasks such that the model is trained with a pair of data and labels and outputs a real-value from a range. In the context of imaging data, this could be used to extract numerical features from an image. Xie et al. and Xue and Ray show promising results for using convolutional neural networks to extract numerical features from images of cell by using neural networks as a regression model that count the number of cells in an image (\cite{https://arxiv.org/pdf/1708.03307.pdf, https://www.robots.ox.ac.uk/~vgg/publications/2015/Xie15/weidi15.pdf}).

\section{Finding structure in high-dimensional data}

The data we will be studying consists of images of cells obtained through high content screening (HCS). HCS is a method for capturing images of cells in multi-well plates, using high-resolution microscopy (\cite{https://www.ncbi.nlm.nih.gov/pubmed/23035272}). A plate captured with high content screening can yield a large number of images in very high-resolution, e.g. 2000x2000 pixels. This makes the analysis of the physical characteristics of a cell possible at a granular level. However, this also makes the dataset high-dimensional which requires the use of high-dimensional visualisation techniques if visualisation is required.

The following section highlight two commonly used techniques for high-dimensional data visualisation.

%High content screening (HCS) is a method for capturing images of cells in multi-well plates, using high-resolution microscopy. It can be used to analyse physical characteristics of the cells captured on images. HCS is a choice method for analysing how compounds can alter cells from images, i.e. drug discovery \cite{https://www.ncbi.nlm.nih.gov/pubmed/23035272}.

\subsection{t-SNE and UMAP}
t-distributed stochastic neighbor embedding (t-SNE) was developed in 2008 by van der Maaten and Hinton as a technique to map high-dimensional data to two- or three-dimensional space. t-SNE can find structure in high-dimensional data points by using the local relationships between data points and optimising results using gradient descent. [how much detail to go into? don't want to go off tangent]%These local relationships are defined using a Gaussian probability distribution in high dimensional space, and then recreated using the Studnet t-distribution.

Uniform Manifold Approximation and Projection for Dimension Reduction (UMAP) is a dimensionality reduction technique which was published in 2018 and has shown competitive results compared to t-SNE. UMAP works by constructing a high-dimensional weighted graph representation of the data. Each edge between points in the graph is weighted according to how likely the points are to be connected. UMAP transforms the high-dimensional graph representation into a low-dimensional representation that is as similar as possible optimising results like t-SNE does (\cite{https://pair-code.github.io/understanding-umap/}).

[how much of the paragraphs below to include?]

The main differences between t-SNE and UMAP are of speed and parameters. The original UMAP paper compares UMAP's performance with t-SNE's on the full MNIST digit dataset which consists of 70,000 images of 784 pixels. On a 2017 Macbook Pro with i7 Core and 8GB of RAM, UMAP takes 87 seconds to run, while t-SNE takes 1,450 seconds (\cite{https://arxiv.org/pdf/1802.03426.pdf}).

t-SNE's main parameter to be tweaked is `perplexity', which loosely corresponds to a guess of the number of neighbours a data point has (\cite{https://distill.pub/2016/misread-tsne/}). UMAP's main parameters are number of neighbours and minimum distance. The former corresponds to the number of approximate neighbours a data point has, similar to t-SNE's perplexity. The latter corresponds to the minimum distance between points in low-dimensional space, meaning that it will tell UMAP how tightly to cluster points together, making visualisation more flexible.

\subsection{Example visualisations}

[unsure about keeping this section]

The capabilities of t-SNE and UMAP are best illustrated through examples.

[mammoth 3D to 2d example]
[fashion mnist data]

In the case of our research, applying t-SNE and UMAP to a dataset of microscopic image could allow us to uncover whether or not immune cells behave in recognisable ways under different experimental conditions and whether or not this can be recognised from the structure of the images.

\section{The place of Deep Learning in Immunology}

There is a number of existing research that uses broader Machine Learning techniques in the field of immunology. Muh et al. (\cite{https://www.ncbi.nlm.nih.gov/pubmed/19516900/}) applied Support Vector Machines (SVMs) to the study of allerginicity. Allergic reactions are triggered when a immune system wrongly assumes a harmless substance to be dangerous, such as dust, and produces antibodies to attack it  (\cite{https://www.immunology.org/policy-and-public-affairs/briefings-and-position-statements/allergy}). The SVMs were used to analyse the DNA sequences of known allergens and known non-allergens. The aim was to try and make accurate predictions on unseen sequences and classify them as either allergenic or non-allergenic. The model achieved 95.3\% accuracy.  In another classification example, MP et al (\cite{https://www.ncbi.nlm.nih.gov/pubmed/20144194/}) used a Bayesian classifier and a decision tree to predict the likelihood of degenerative disorders from the sequencing of antibodies and achieved a best accuracy score of 89\%.

The above examples highlight a sample of immunology research carried out using traditional Machine Learning techniques. In the specialised field of deep learning, research using immune cell data is mostly focused on cell counting (Turkki et al., 2016 \cite{https://www.ncbi.nlm.nih.gov/pmc/articles/PMC5027738/}; Aprupe et al., 2019 \cite{https://www.ncbi.nlm.nih.gov/pmc/articles/PMC6462181/}). Broader deep learning research on cell imaging data also includes cell segmentation (Al Kohafi et al., 2018). On the other hand, in cancer research, deep neural networks are increasingly being used for feature extraction from images to accurately detect cancer (Litjens et al., 2016 \cite{https://www.ncbi.nlm.nih.gov/pmc/articles/PMC4876324/}; Bychkov et al., 2018; \cite{https://www.ncbi.nlm.nih.gov/pmc/articles/PMC5821847/}).

\section{Summary}

The research cited here highlights that there is an array of methods available to process high-dimensional, visual data through deep learning and visualisation techniques. This section has also shown that immunology-related fields such as cancer research have successfully applied deep learning methods in their research to obtain promising results, and that immunology researchers have also successfully made use of machine learning techniques. There is an increasing interest in the applications of Deep Learning in medical fields, but the use of deep neural networks has not been fully explored in the context of studying immune cells interaction. There seems to be a lack of research into the qualitative and quantitative analysis of immune cell interactions from imaging data through deep learning. However, we have shown that immune cell interactions are of particular interest in drug research as they are key in understanding how an immune system operates. %Indeed, deep neural networks have been used in immunology for the purposes of cell segmentation and cell counting.

This dissertation will thus focus on filling this gap by using deep learning to extract features from images of cells in order to uncover qualitative or quantitative data about the interactions between types of immune cells under different experimental conditions. Subsequently, the results of this research could show whether this shows promise and should be explored further in the future.


\chapter{Materials and Methods} \label{sec:mm}

This chapter covers the image materials that were available to analyse, how they were processed, and which methods were applied to them. The details of the implementation of these methods is then discussed in Section \ref{sec:implementation}.

\section{Immune cells dataset}

\subsection{Setup}

The images that were used for the purpose of this research were provided by a researcher in Immunology at the University of Glasgow. The images were captured from multi-well plates with a commercial INCell Analyzer Machine\footnote{https://www.gelifesciences.com/en/us/shop/cell-imaging-and-analysis/high-content-analysis-systems/instruments/in-cell-analyzer-2500-hs-high-content-analysis-hca-imaging-system-p-04586}. As established in Section \ref{bg:immunesystem}, the type of immune cells we are studying are T-cells and dendritic cells (DCs). Each plate to be imaged in the INCell Analyzer Machine contains a grid of wells. Each well is assigned a label and an experimental condition. T-cells, dendritic cells, and compounds related to the experimental conditions are injected in the well. For distinction, the cells are loaded with fluorescent dyes: the T-cells are dyed with a green dye (FITC dye), and the dendritic cells are dyed with a red dye (TexasRed dye). After imaging, we obtain three field-of-view images per well:

\begin{itemize}
    \item a Brightfield image, which shows both T-cells and dendritic cells (Figure \ref{fig:fov_brightfield})
    \item an image showing only the T-cells, which has been captured thanks to the fluorescent green dye (Figure \ref{fig:fov_fitc})
    \item an image showing only the dendritic cells, which has been captured thanks to the fluorescent red dye (Figure \ref{fig:fov_tr}).
\end{itemize}

\begin{figure}[h]
    \centering
    \begin{subfigure}[h!]{0.3\textwidth}
        \includegraphics[width=\textwidth]{dissertation/figures/example_Brightfield.png}
        \caption{Brightfield view}
        \label{fig:fov_brightfield}
    \end{subfigure}
    \begin{subfigure}[h!]{0.3\textwidth}
        \includegraphics[width=\textwidth]{dissertation/figures/example_FITC.png}
        \caption{Green dye (T-cells) view}
        \label{fig:fov_fitc}
    \end{subfigure}
    \begin{subfigure}[h!]{0.3\textwidth}
        \includegraphics[width=\textwidth]{dissertation/figures/example_TexasRed.png}
        \caption{Red dye (DCs) view}
        \label{fig:fov_tr}
    \end{subfigure}
    \caption{Square patches from microscopic field-of-view images, 33.3x zoom}
    \label{fig:fov}
\end{figure}

\subsection{Experimental conditions}

[need some explaining about labels and drugs here]

\subsection{Picking images}

There was a large amount of images available from different well plates with different experimental conditions. However, each set of images represents about 8GB of data on average. Moving images through disks or cloud filing system represented substantial time and was vulnerable to transfer errors. Hence, a limited number of plates were picked out for training and evaluation to make sure their consistency could be validated. The plates chosen had to be picked keeping in mind the experimental conditions they represented.

\begin{itemize}
    \item The ``DMSO" dataset: [TODO to be read over by Hannah] DMSO is a solvent that helps solubilise the drug compounds in a well as most compounds are not initially water soluble. The drug compounds being more soluble, they should then be able to have more of an impact.
    \item The ``balanced” dataset: this dataset contains an equal number of images in the three categories of stimulation: no drug simulation, stimulation with OVA peptide, and stimulation with ConA. This is to fight issues of class imbalance when training the model.
    \item The simpler dataset, with two categories: this  dataset contains an equal number of images in two categories: no drug simulation, and simulation with OVA peptide. This was picked in the hope that if no results are obtained with 3 categories, a model might be able to perform better with two.
\end{itemize}

\subsection{Pre-processing}
The datasets obtained from this setup consisted of 2048x2048 12-bit images in TIFF format. As mentioned above, each ``image” consists in fact of a set of three field-of-view images: the T-cells, the dendritic cells, and the Brightfield image. The Brightfield images were used to gain an overview of what the well might look like, but were discarded and not used for analysis.

%Fiji (ImageJ) was used to explore the images as the particular 12-bit format of the image in a TIFF file meant that standard image previewers reproduced the image as all black. Fiji also offered useful tools for testing image pre-processing methods, such as binary filters, thresholding, and background correction.
Each of the images sized about 8MB, and represented 4,194,304 pixels. Each plate had about 400 wells, which corresponds to 800 images when counting both the T-cell image and the DC image. This represented an issue of very high dimensions to deal with, and little images to feed into any kind of model.

Moreover, Figures \ref{fig:fov_fitc} and \ref{fig:fov_tr} shows that even to the naked eye, the smaller white dots could easily be confused for dust on the screen, and could be as confusing for a deep learning model trying to learn features from an image as they are for us. Furthermore, they remained of very high dimensions. A basic autoencoder with three 2x2 Pooling operations would yield around 500,000 pixel points, which is a very high number for a visualisation technique like t-SNE or UMAP.

\bigskip
\subsubsection{Sliding window}

\hfill\\
\hfill\\
To resolve this issue, the first idea was to make the images more palatable by a neural network by cutting up a set square subsection of the image of smaller dimensions, e.g. 250x250. However, this would still leave an issue of having limited input to train a neural network. Instead, images were pre-processed by passing a sliding window over the image, creating a patches of images per file. This quickly expanded the size of the dataset, making it as big as 58,000 samples in some cases. Smaller images also made more sense to the naked eye, hence the assumption was made that a trained neural network would perform better on this gridded dataset than on a full-image dataset.

\bigskip
\subsubsection{Noise detection and normalisation}

\hfill\\
\hfill\\
From analysing the images, it was also found that some images contained a lot of pixels in the range of [0, 255] before normalisation. Moreover, these pixels seemed to be background noise as shown in Figure \ref{fig:bgnoise}. The immunology researcher providing the images confirmed that this noise was present in some images read from specific plates. Wells in early experiments were injected with additional cells as researchers thought this made the cells `happier', however this was found to make no change. These cells were not dyed, but some of their details still came through the imaging system. Although this background was not always visible to the naked eye, we did not want those pixel values to confuse a neural network model. Hence, values below 255 in images were clipped to 255.

\begin{figure}[h!]
    \centering
    \begin{subfigure}[h!]{0.99\textwidth}
        \includegraphics[width=\textwidth]{dissertation/figures/background_noise_true.png}
        \caption{Histogram and image analysis for a sub-image with noisy cells in the background}
        \label{fig:bgnoisetrue}
    \end{subfigure}

    \begin{subfigure}[h!]{0.99\textwidth}
        \includegraphics[width=\textwidth]{dissertation/figures/background_noise_false.png}
        \caption{Histogram and image analysis for a sub-image with no noisy cells in the background}
        \label{fig:bgnoisefalse}
    \end{subfigure}

    \caption{Histogram and image analysis for background noise detection}
    \label{fig:bgnoise}

\end{figure}

Each sub-image was then normalised with min-max normalisation (\autoref{equation:minmax}) to get a [0,1] range of pixel values.

\begin{equation}
    minmax(x) = \frac{x - min(x)}{max(x) - min(x)}
\label{equation:minmax}
\end{equation}

\bigskip
\subsubsection{Outlier detection}

\hfill\\
\hfill\\
\begin{figure}[h]
    \centering
    \begin{subfigure}[h!]{0.3\textwidth}
        \includegraphics[width=\textwidth]{dissertation/figures/faulty_brightfield.jpg}
        \caption{Brightfield view}
    \end{subfigure}
    \begin{subfigure}[h!]{0.3\textwidth}
        \includegraphics[width=\textwidth]{dissertation/figures/faulty_tcell.jpg}
        \caption{Green dye (T-cells) view}
        \label{subfig:tcell}
    \end{subfigure}
    \begin{subfigure}[h!]{0.3\textwidth}
        \includegraphics[width=\textwidth]{dissertation/figures/faulty_dcell.jpg}
        \caption{Red dye (DCs) view}
        \label{subfig:dc}
    \end{subfigure}
    \caption{Different views of an image which contains ``Faulty" patches. \protect\subref{subfig:tcell}, \protect\subref{subfig:dc} have been brightness-adjusted for visualisation.}
    \label{fig:noisyimage}
\end{figure}
As described above, the provided images sometimes contained some background noise. Furthermore, some images contained larger amounts of noise coming from defects in the well such as water droplets. This is illustrated in Figure \ref{fig:noisyimage}. Their pixel value distribution followed the [0, 255] range as described above but these values covered the whole sub-image and no cells of interest were present in those patches. Removing them entirely from the dataset made it more difficult to reason about whole images, as a full image is represented by 100 sub-images, and removing them would create inconsistencies in the amount of sub-images per image file. Instead of removing them, it was decided to keep them in to see if a neural network could make sense of them as a category. These were labelled as ``Faulty".

\bigskip
\subsubsection{Labelling}

\hfill\\
\hfill\\
Each plate came with an Excel sheet giving information about the plate layout. Each image was given a letter and a name, and the Excel sheet gave information about drug stimulation, compound ID number, or compound concentration. These Excel sheets were not automatically parsable as types of drugs or location in the sheet might vary from one to the next, hence labelling had to be hardcoded and handchecked.

\subsection{Combining images to qualify interaction} \label{subsec:combining}

To summarise, we have established that for each well representing an experimental condition, we obtain two images of interest: an image of T-cells, and an image of dendritic cells. While these are obtained separately with the help of fluorescent dyes, they are still captured from the same well in which they are placed together in. Hence, for us to gain any understanding of cell interaction from these images, we need to combine them one way or another.

We decided to combine each black-and-white T-cell and DC image in one RGB image. Computationally speaking, a RGB image corresponds to a multi-dimensional array of three arrays. Each of these arrays corresponds to a colour channel: red, green, and blue. The dendritic cell image has been obtained through fluorescent red dye screening, so the red channel of the image is set to this image. Similarly, the T-cell image has been obtained through fluorescent green dye screening, so the green channel of the image is set to this image. The blue channel of the RGB image is left blank. This RGB image allows us to visualise T-cells in green, DCs in red, and any close overlap between those cells will be in orange hues. Figure \ref{fig:combined} illustrates a sample of combined sub-images after this operation is completed.

This combination allows us to visualise areas of proximity in cells as well as areas of overlap. This visualisation is a means of qualifying interaction between the different types of immune cells.

\begin{figure}[h]
    \centering
    \includegraphics[width=\textwidth]{dissertation/figures/combined_cells.png}
    \caption{Random sample of five RGB images, where the red channel contains the dendritic cells and the green channel contains the T-cells.}
    \label{fig:combined}
\end{figure}

\section{Image segmentation}

Image segmentation refers to the process of separating out the parts of interest of an image into different `segments' or objects. In our context, applying image segmentation to our dataset had two purposes. The first was to use image segmentation as a method to separate the background from the cell objects for background correction. The second was to use it to obtain binary objects of the cells in order to compute numerical data about cells in the images.

\subsection{Background correction} \label{subsec:correction}

A common issue for microscopic images obtained systemically through different screening systems is that of a noisy background [ref]. This can usually be corrected through different methods, such as flat field correction [ref]. Flat field correction removes image noise by using a ``neutral” background image without any additional objects (i.e. cells) as a reference image. However, the provided dataset did not come with a flat field image. As such, alternatives had to be explored.

All the images in the original, uncombined dataset are black and white, with the details of interest (the cells) in bright white spots. However, as discussed in Section \ref{subsec:preproc}, the image can contain noise coming from gray details in the image that the naked eye cannot see immediately, but that could influence how a model learns. Hence, we need a method that will separate out the cell pixels from the background. We can achieve this by obtaining a binary mask of the image, such that the white pixels of the binary image correspond to the cells, and the dark pixels correspond to the background. There are two ways of doing this which we will describe further in Section \ref{sec:implementation}: K-means segmentation, and thresholding.

Once a binary mask is obtained through these methods, we can remove the background of the original image by multiplying it with the mask, a procedure known as \textit{masking}. If the binary mask is satisfactory, the output of this image should only contain the cells, who will have kept their detail, while the background will have been blacked out.

\subsection{Quantifying interaction}

The process used in \autoref{subsec:combining} describes a way of visually qualifying interaction between immune cells. However, we are also interested in quantifying interaction and obtaining a numeric value for immune cell interaction observed in an image.

We can reuse the method described in subsection \ref{subsec:correction} for obtaining the mask of the cell objects. This mask can be used for further calculations. A common metric for evaluating image segmentation quality is Intersection over Union (IoU), also known as the Jaccard index [ref].

\begin{equation}
    IoU(X,Y) = \frac{X \& Y}{X | Y}
\end{equation}

In this case, we can use the IoU as a metric for area of overlap between two separate cell objects: the T-cell objects and the dendritic cell object obtained from the same image in the same experimental conditions. We can use the concept of overlap to quantify the level of interaction between cells. This number will then be used as a label input to a deep regression model. The aim will be to evaluate whether we can train a model to recognise a numeric value of interaction from an image.

\begin{figure}
    \centering
    \includegraphics[width=0.8\textwidth]{dissertation/figures/mask_overlap_operation.png}
    \caption{Example of visual and numerical overlap between two sets of immune cells}
    \label{fig:mask_overlap}
\end{figure}

\section{Autoencoder}

As described in Section \ref{sec:research}, autoencoders are a particular type of neural network built with symmetrical layers around a bottleneck. The aim of an autoencoder is to map an input to its output as close as possible while reducing its dimensions. The hope is that if an image is reduced to a certain number of dimensions, and a neural network is able to reconstruct the original image very closely just based on that compressed representation, then that smaller code representing image is a good enough representation of the original input that can be fed into other models or algorithms.

[autoencoder figure here?]

The following two sections highlight the purposes of developing an autoencoder in the context of our research.

\subsection{Visualising high dimensional data}

The first use for our autoencoder is to reduce the dimensions of our image input, as the size of the data points will not only make it slow for data visualisation techniques to process, but also harder for them to distinguish any structure in the data. High dimensionality visualisation techniques such as t-sne and UMAP can help visualise if there is an inherent structure to the data. We want to use this as a tool for analysis of immune cells interactions. We want to evaluate whether or not a technique such as t-SNE or UMAP can cluster different images around similar experimental conditions. If such successful groupings were found, it would allow us to show that different drug compounds yield structurally similar cell interactions. The aim is to be able to qualify interaction from an image.

\subsection{Quantifying interaction in unseen images}

The second user of our autoencoder will be as a building block for a deep regression model. If our full autoencoder shows a satisfactory reconstruction performance, then we can use the encoder block of the model and extend it in a regression structure. This autoencoder-based regression model will take images and associated overlap values as input for training. We will then use the model to make predictions for overlap values on unseen images and assess whether or not it does so successfully. The aim is for our regression model to be able to quantify interaction from an image.



\chapter{Implementation} \label{sec:implementation}

\section{System diagram}

Figure \ref{fig:system} can help the reader to gain understanding how each of the materials and methods used as described in Section \ref{sec:mm} fit together.

\begin{figure}[h!]
    \centering
    \includegraphics[width=\textwidth]{dissertation/figures/system_diagram.pdf}
    \caption{[TO ANNOTATE] System diagram showing how each image will be decomposed and analysed.}
    \label{fig:system}
\end{figure}

\section{Experimental setup}

As this is a research project based on using deep learning methods, a large part of the research involved an experimental process of tweaking the deep learning models, training them, and evaluating them.

\section{Pre-processing}

Firstly, before feeding them as input into any model, the images of cells had to be pre-processed. This is already partly discussed in Section \ref{}. This section will go over the implementation specifics.

\bigskip
The following list provides a step-by-step of the process:
\begin{enumerate}
    \item For each dataset, all the images of T cells and dendritic cells were picked out. Their corresponding filenames were retrieved.
    \item The filenames were sorted to keep images from the same well next to each other in memory.
    \item Each image was partitioned into sub-images through a sliding window approach. The window was of size 192x192 and yielded a 100 same-sized images. The original image was 2048x2048 so the right and bottom borders of the image were discarded.
    \item Once this gridded dataset was obtained, each T cell image was combined in a RGB image with its dendritic cell counterpart, stored in index i and index i+100 respectively.
    \item For each combined image, noise was removed through low value pixel clipping. Min-max normalisation was then applied to put all images in the same [0,1] range.
\end{enumerate}

Some implementation choices made in the above have to be justified.

\subsection{Sliding window size}

The sub-image window size was chosen to be 192x192 because of the way Pooling and Upsampling operations function in autoencoders. Autoencoders are built around symmetrical operations around a bottleneck. That is, we are trying to output a reconstructed image which will be of the same size as the input.

[make an image]
%Encoder layers in autoencoders make use of Pooling operations to reduce dimensionality of the input as you go deeper into the network. These pooling operations are commonly done in factors of 2 for their efficiency at downsizing as well as maintaining detail. In the context of an image passing through an autoencoder, it is reduced by 2 for every pooling operation. Each pooling operation then has a corresponding upsampling operation in the decoder part of the autoencoder, in order to reconstruct the image with the right size. An upsampling operation of factor 2 outputs an image of dimensions multiplied by 2. However, a pooling operation will round up resultant dimensions if the reduction of the original dimensions results in a decimal number. The corresponding upsampling operation would then double that number, which would result in a different dimension in the output of the autoencoder. Hence, we had to pick a size that would be big enough to show enough cells, but we also had to pick a size which meant the image could be divided by 2 for a large amount of times, depending on the number of hidden layers. To illustrate, if we apply a division by to the number 200 three times, the image can be upsampled and reconstructed correctly, however if it is reduced 4 times, then after the 4th operation we obtain a 13x13 image and the first upsampling operation creates a 26x26 image. A size which is a multiple of 8 bypasses that issue for a larger number of operations. A 192x192 image can be reduced 6 times, when it reaches dimensions of 3x3, without running into upsampling issues.

\subsection{Image combination}

Images of each type of cells on their own would not make much sense on their own when the aim is to quantify the level of interaction. Hence they have to be combined. The advantage that the provided dataset has is that it comes with the T cells and the dendritic cells separated by means of fluorescent dyes, hence no image segmentation technique had to be used to be able to separate the image out into the different types of cells and colour them in. Instead, each of the T cell and its corresponding DC, which were associated to the same file, were combined together in an RGB image, with the blue channel set to 0. %The images were not combined in an absolute difference operation as we would have lost information of where the images overlapped i.e. interacted, which is what we are looking into in this research.

\section{Quantifying immune cell interaction}

% To obtain an interaction measure from the images, multiple methods were explored. First, we looked at the Structural Similarity Index (SSIM) of images. was computed, however because the images were so similar in general i.e. white blobs spread across a black background, the results were extremely similar across all cells in different experiment conditions.

The idea was [instead] to obtain masks of the images and compute the intersection over union area as the interaction value. As the cells were bright blobs on a black background, and we did not have the task of separating T-cells from dendritic cells, it was hoped that this would be straightforward. Both K-means and thresholding methods yielded good results and their specifics are detailed below.

\subsection{\textit{k}-means colour clustering}

k-means has been shown to perform well on image segmentation by quantising the number of colours in an image into \textit{k} clusters. Formally, k-means aims to partition data points in an array into \textit{k} sets such that the variance between points within clusters is minimised. In our case we wanted to use k-means to transform our black-and-white images of immune cells into bichrome images that we could use as masks. [The following pseudocode details the process of K-means.]

\begin{algorithm}
    \DontPrintSemicolon
    \KwData{$I$, an array of pixel values making an image.\;
    $k$, the number of colours to partition the image's colour palette to.\;}
    \KwResult{A set of $k$ clusters.}
    \Begin{
        Initialise $k$ objects picked from $I$\;
        \While{clusters are still changing}
        {
           Assign each item $i$ in $I$ to the cluster with closest mean value\;
           Recompute the mean of each cluster\;
        }
    }

\caption{Pseudocode for the k-means algorithm applied to image segmentation.}
\label{alg:kmeans}
\end{algorithm}

k-means clustering is conveniently offered by multiple libraries in Python. We looked at both scikit-learn's and OpenCV's k-means. scikit-learn is a general library  for machine learning tools, while OpenCV is a more specialised library built for Computer Vision purposes. Both their k-means functions are straightforward to initialise and use. Their performance was benchmarked in order to select the best one. The table below reports times for k-means initialised with k=2, random initialisation of centroids and 10 iterations. The test was ran on a 2015 MacBook Pro with 2.7 GHz i5 core and 8 GB memory.

% Please add the following required packages to your document preamble:
% \usepackage{multirow}
\begin{table}[]
\centering
\begin{tabular}{|l|l|l|}
\hline
\textbf{Initialisation} & \textbf{OpenCV} & \textbf{Scikit-learn} \\ \hline
Random                                   & \multicolumn{1}{r|}{18.9s}       & \multicolumn{1}{r|}{165s}              \\ \hline
k-means++                                & \multicolumn{1}{r|}{29.3s}       & \multicolumn{1}{r|}{139s}              \\ \hline
\end{tabular}

\caption{CPU times for OpenCV's and scikit-learn's k-means tool ran on 1,000 samples of 192x192 pixels with different methods of initialising centroids.}
\end{table}

As we can see OpenCV outperforms scikit-learn in all cases. OpenCV for Python is a wrapper library around the original OpenCV code built in C++, which gives it a boost in performance. OpenCV's k-means was thus selected. Initially, k-means centers were initialised randomly. However, during training and validation it was found that this method of initialisation was yielding highly different results for the intersection over union metric at every run. Hence, some speed was traded for consistency and the kmeans++ center initialisation method was picked instead.

\subsection{Thresholding}

An alternative to k-means in the case of black-and-white images is thresholding. We decided to explore this option as it could have performance improvements compared to K-means.

Thresholding refers to the process of converting a grayscale image to a binary image of pixels. Pixels above a set threshold are set to 1, and the rest of the pixels below that threshold are set to 0. Thresholding depends on pixel distribution analysis. Usually, thresholding works well for images which have different peaks of pixel values in their distribution. However, in the case of our images we had one peak of pixel values. Figure \ref{fig:thresholdhist} illustrates this.

\begin{figure}[h]
    \centering
    \begin{subfigure}{0.45\textwidth}
        \centering
        \includegraphics[width=.5\textwidth]{dissertation/figures/sample_grayscale.jpg}
    \end{subfigure}
    \begin{subfigure}{0.45\textwidth}
        \centering
        \includegraphics[width=.5\textwidth]{dissertation/figures/sample_cell.jpg}
    \end{subfigure}
    \begin{subfigure}{0.45\textwidth}
        \centering
        \includegraphics[width=.5\textwidth]{dissertation/figures/grayscale_histogram.png}
    \end{subfigure}
    \begin{subfigure}{0.5\textwidth}
        \centering
        \includegraphics[width=.45\textwidth]{dissertation/figures/cell_histogram.png}
    \end{subfigure}
    \caption{Example images and their histogram. As we can see the grayscale image of lillies has two peaks of frequency.}
    \label{fig:thresholdhist}
\end{figure}
As such, we had to identify an alternative for a threshold. First, we selected the mean pixel value as the threshold. This yielded acceptable results, however some noisy pixels still came through the mask (see Figure \ref{fig:thresholdmean}). To fix that problem, the threshold value was set as the sum of the mean pixel value and the standard deviation. This decision was based on the hypothesis that the noise level of an image with a flat structure can be estimated from its variation. Results were satisfactory, as shown in Figure \ref{fig:thresholdstd}.

\begin{figure}[h]
    \centering
    \begin{subfigure}[h!]{0.4\textwidth}
        \includegraphics[width=\textwidth]{dissertation/figures/mean_threshold_cell.jpg}
        \caption{Threshold: mean pixel value}
        \label{fig:thresholdmean}
    \end{subfigure}
    \begin{subfigure}[h!]{0.4\textwidth}
        \includegraphics[width=\textwidth]{dissertation/figures/mean_std_threshold_cell.jpg}
        \caption{Threshold: mean + standard deviation}
        \label{fig:thresholdstd}
    \end{subfigure}
    \caption{Segmented images according to different threshold values.}
\end{figure}

%\subsection{Image masks for background correction}

%As mentioned in Section \ref{subsec:correction}, the microscope images of immune cells might have noise, especially in the background. Particularly in biomedical data, there has been research on background correction. Rather than fixing the background, we can evaluate whether removing the background entirely helps in this case by evaluating both masked and unmasked images.

\section{Convolutional autoencoder}

The main tool to be developed to exploit this dataset was a convolutional autoencoder. The autoencoder was built for two purposes: obtaining a smaller "code" representing each of the images to be fed into high-dimensional visualisation algorithms, and to be the starting block for a deep regression model.

\subsection{Structure}

The autoencoder was built using Keras \footnote{https://keras.io} for Python. We based our architecture off the standard Convolution --> Activation --> Pooling sequence of operations commonly used in convolutional neural networks (ref). The aim was to maximise the reduction of dimensionality while maintaining a satisfactory reconstructed image. Hence, the choice of number of hidden layers had to be made as a compromise.

The autoencoder was tuned by evaluation on a training and validation dataset. Its structure was established through both literature review and trial and error. The choice of hidden layer activation is PReLU, because of its evidenced benefit in improved loss (ref), as well as showing slightly better results in training. Moreover, convolutional layer sizes were kept quite high, instead making the neural network deeper to reduce dimension. Strides are used in the last layer instead of max pooling (why?) because results were slightly better. The choice of feature maps being higher in earlier layers comes from the encoded representation being smaller this way, as well as differences with the reversed being marginal. Finally, the output layer activation is sigmoid, as we are trying to predict a value between 0 and 1, as the images have been normalised. 

\subsection{Deep regression}

The regression model was built on the encoder layers of the autoencoder. The structure of the regression layers of the model was kept simple. Only two fully connected layers are used, with a Dropout layer in between. Dropout has shown to make models more robust and prevent overfitting.
Both softplus and linear activations were tried for the regression model. The linear activation was accompanied with a kernel restriction on the keras model of non-negativity, as interaction cannot be negative. Softplus keeps its output values positive. The results were similar, however the linear function performed overall better.


\chapter{Evaluation} 

\section{Train-test dataset split}

A large amount of data was available to us. In order to best evaluate the performance of our trained autoencoder and regression models, part of the data was picked for training, and the rest was left out for testing. Part of the training data was used while training for validation and model checkpoints.

As explained in Section \ref{}, three datasets were picked out for research. Training was done on the balanced dataset. The balanced dataset was not prone to class imbalance issues, and contained an equal mix of images from each class: unstimulated immune cells, immune cells stimulated with OVA, and immune cells stimulated with ConA.  

Part of this balanced dataset was kept untouched for testing. Two other datasets were used for testing: the DMSO dataset, and the two-category dataset. 

The two-category dataset had two purposes: evaluating if the model performed better with differentiating between two classes instead of 3. Moreover the two-category dataset contained both labels of drug stimulation and drug concentration. It was hypothesised that maybe we could gather more information of interaction based on drug concentration rather than type of drug. 

The DMSO dataset had issues because it was built from the balanced and two-category dataset and parts of the DMSO images were possibly used to train the balanced dataset. However, DMSO represents the conditions in which difference in interaction should be most visibile, thus it was still valuable to evaluate for results. However, it is considered "tainted". 


%\section{Autoencoder}
%\subsection{Reconstruction}
%\subsection{2D Visualisation with t-sne and UMAP}
%\subsubsection{Matplotlib tool for outlier exploration}
\chapter{Conclusion}    

The purpose of our research was to apply deep learning methods to microscope images of immune cells and evaluate whether a deep learning approach could be successfully applied to the analysis of interaction between different types of immune cells. 

To assess this, we implemented a convolutional autoencoder from which we implemented a regression model. We wanted to use the autoencoder's power at dimensionality reduction to visualise the structure of the imaging data, as well as use it to build a powerful regression model capable of predicting a value of interaction from an image of immune cells. The specific questions we were looking to answer were whether or not there was an underlying structure to the images of immune cells under different experimental conditions, and whether or not we could quantify interaction from an image of immune cells.

Our evaluation of these two models showed that there is potential in using deep learning methods for imaging data of immune cells, following in the footpath of similar research done in cancer research. While we could not successfully find an inherent structure to the data using a UMAP projection, this could have been influenced by the choice of datasets and size for creating patches from the raw images. Moreover, we showed that a regression model could predict a value of overlap from an image with a best RMSE score of $1.161 \pm 4.732$ on a background-corrected dataset containing images from all our categories. 

With appropriate pre-processing and further research, deep learning techniques could be a new approach to the analysis of interaction between immune cells, allowing researchers to analyse their datasets further.

\section{Future work}

There is a number of different routes that could still be explored. Firstly, we only explored one size of sub-images in our pre-processing. A bigger size of sliding window could include more details of the images and allow for a better global overview of the impact of experimental conditions on immune cells. As such, bigger sub-images might reveal a structure that an algorithm such as UMAP could analyse fast with the help of an autoencoder.

Furthermore, we only evaluated one metric for our regression task, which was the percentage of overlap represented by the intersection-over-union metric. This represents some limitations in simplicity of analysis. Indeed, a large clustering of T cells around a dendritic cell \textit{without overlap} could also signify a level of interaction, without overlap being observed. Moreover, we are using a two-dimensional view of the cells. Two different types of cells overlapping might not mean that they are interacting. A T cell could simply be sitting on top of a dendritic cell, without communication happening between the two.

Finally, we have briefly touched upon the U-Net model. Our dataset gave us access to pre-segmented pictures of immune cells. However, we could use this wealth of pre-segmented data to train a U-Net model to segment T cells and dendritic cells object in greyscale images. Alternatively, such a model could be used to extract features from segmented immune cells, such as size of cells and granularity. 

%==================================================================================================================================
%  APPENDICES  

\begin{appendices}

\chapter{Appendices}

\section{Autoencoder model initialisation}

\begin{lstlisting}[language=python, float, caption={Keras code for initialising the autoencoder model developed here. It contains 5 downsampling and upsampling operations. It returns both its encoder and decoder parts. The decoder is used to evaluate the performance of the autoencoder at image reconstruction. The encoder is used to encode images to project them onto a two-dimensional plane using t-SNE and UMAP, and is also the building block for our regression model. Started from a tutorial by \citet{chollet_keras}, and was expanded through experiments and research.}, label=lst:autoencoder]
    def make_autoencoder():
        """
        Initialise autoencoder model for training and return reference to both decoder and encoder parts of the model.
        """
    
        # image shape is defined in the configuration
        input_img = Input(shape=(imw, imh, c))
    
        # layers for reduction of image
        x = Conv2D(64, (3, 3), padding='same')(input_img)
        x = PReLU()(x)
        x = MaxPooling2D((2, 2), padding='same')(x)
        x = Conv2D(32, (3, 3), padding='same')(x)
        x = PReLU()(x)
        x = MaxPooling2D((2, 2), padding='same')(x)
        x = Conv2D(32, (3, 3), padding='same')(x)
        x = PReLU()(x)
        x = MaxPooling2D((2, 2), padding='same')(x)
        x = Conv2D(32, (3, 3), padding='same')(x)
        x = PReLU()(x)
        x = MaxPooling2D((2, 2), padding='same')(x)
        x = Conv2D(32, (3, 3), padding='same', strides=2)(x)
        x = PReLU()(x)
        
        # bottleneck layer
        encoded = Flatten()(x)
    
        # layers for expansion of image 
        x = UpSampling2D((2, 2))(x)
        x = Conv2D(32, (3, 3), padding='same')(x)
        x = PReLU()(x)
        x = UpSampling2D((2, 2))(x)
        x = Conv2D(32, (3, 3), padding='same')(x)
        x = PReLU()(x)
        x = UpSampling2D((2, 2))(x)
        x = Conv2D(32, (3, 3), padding='same')(x)
        x = PReLU()(x)
        x = UpSampling2D((2, 2))(x)
        x = Conv2D(64, (3, 3), padding="same")(x)
        x = PReLU()(x)
        x = UpSampling2D((2, 2))(x)
        decoded = Conv2D(c, (3, 3), activation='sigmoid', padding='same')(x)
    
        decoder = Model(input_img, decoded)
        encoder = Model(input_img, encoded)
        # the encoder will be trained through the decoder so it does not need to be compiled
        decoder.compile(optimizer='adam', loss='binary_crossentropy')
    
        return decoder, encoder
\end{lstlisting}

\section{Regression model initialisation}

\begin{lstlisting}[language=python, caption={Keras code for initialising the regression model developed here. It is constructed from an encoder model which has been previously initialised. The encoder is then extended with fully connected (Dense) layers of decreasing sizes, separated by a Dropout layer for robustness. The final layer is activated by a linear function constrained not to be negative, and outputs a numeric value.}, label=lst:regression]
    def make_regression(encoder):
        """
        Initialise a regression model for training using
        a previously created encoder model
        """
    
        model = Sequential()
        model.add(encoder)
        model.add(Dense(128, activation='relu'))
        model.add(Dropout(0.15))
        model.add(Dense(64, activation='relu'))
        model.add(Dense(1, activation='linear', kernel_constraint=constraints.NonNeg()))
    
        model.compile(loss='mean_squared_error', optimizer='adam')
        
        return model
\end{lstlisting}

\end{appendices}

%==================================================================================================================================
%   BIBLIOGRAPHY   

% The bibliography style is abbrvnat
% The bibliography always appears last, after the appendices.

\bibliographystyle{abbrvnat}

\bibliography{l4proj}

\end{document}
