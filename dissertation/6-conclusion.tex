\chapter{Conclusion}    

The purpose of our research was to apply deep learning methods to microscope images of immune cells and evaluate whether or not a deep learning approach could be successfully applied to the analysis of interaction between different types of immune cells. 

To assess this, we implemented a convolutional autoencoder from which we implemented a regression model. We wanted to use the autoencoder's power at dimensionality reduction to visualise the structure of the imaging data, as well as use it to build a powerful regression model capable of predicting a value of interaction from an image of immune cells. The specific questions we were looking to answer were whether or not there was an underlying structure to the images of immune cells under different experimental conditions, and whether or not we could quantify interaction from an image of immune cells.

Our evaluation of these two models showed that there is potential in using deep learning methods for imaging data of immune cells, following in the footpath of similar research done in cancer research. While we could not successfully find an inherent structure to the data using a UMAP projection, this could have been influenced by the choice of datasets and size for creating patches from the raw images. Moreover, we showed that a regression model could predict a value of overlap from an image with a best RMSE score of 1.161 $\pm 4.732$ on a background-corrected dataset containing images from all our categories. 

With appropriate pre-processing and further research, deep learning techniques could be a new approach to the analysis of interaction between immune cells, allowing researchers to analyse their datasets further.

\section{Future work}

There is a number of different routes that could still be explored. Firstly, we only explored one size of sub-images in our pre-processing. A bigger size of sliding window could include more details of the images and allow for a better global overview of the impact of experimental conditions on immune cells. As such, bigger sub-images might reveal a structure that an algorithm such as UMAP could analyse fast with the help of an autoencoder.

Furthermore, we only evaluated one metric for our regression task, which was the percentage of overlap represented by the intersection-over-union metric. This represents some limitations in simplicity of analysis. Indeed, a large clustering of T cells around a dendritic cell \textit{without overlap} could also signify a level of interaction, without overlap being observed. Moreover, we are using a two-dimensional view of the cells. Two different types of cells overlapping might not mean that they are interacting. A T cell could simply be sitting on top of a dendritic cell, without communication happening between the two.

Finally, we have briefly touched upon the U-Net model. Our dataset gave us access to pre-segmented pictures of immune cells. However, we could use this wealth of pre-segmented data to train a U-Net model to segment T cells and dendritic cells object in greyscale images. Alternatively, such a model could be used to extract features from segmented immune cells, such as size of cells and granularity. 