\chapter{Introduction}

\section{Motivation}

Your immune system is your protective shield against pathogens. It functions by discriminating between what is part of your self, and what is not, and fighting what is alien to it.

Now picture your immune system as a speed date. Your immune cells go on quick dates with other immune cells, who tell them about their life. Immune cells might be under the influence of certain substances to various degrees. The success of the discussion between two immune cells during their speed date determines how your immune system is going to evolve as a whole. The consumption of substances might help make the situation more positive, or worse. Your immune system might be pleased with the date, and react positively, or an immune cell might get offended by its date, and trigger a negative response. % These dates could happen very fast with immune cells moving on to their next date, but could also be prolonged into longer relationships.

Indeed, the onset of an immune response in our immune system depends on the interaction strength between different types of immune cells. Certain types of immune cells relay information about their environment to other types of immune cells, which can then trigger an appropriate response depending on what they have learned from the other cell about the environment. The environment might contain substances like alien, dangerous bodies. These interactions between immune cells can be enhanced or inhibited by the application of drugs. Studying the reactions of immune cells under the influence of different types of drugs is key to the development of drugs for diseases such as viral infections, cancer or auto-immune diseases. 

\section{General problem and our idea}

One way of studying the interactions of immune cells is through microscope images obtained in artificial settings. We can place immune cells in a dish and study them under a microscope under different experimental conditions, which could involve different types of drugs being injected into the dish. Microscope images of these cells can then be systematically captured.

We are however not directly interested in how these images are captured, but how they can be analysed. Microscope images are often analysed with proprietary software which is costly to maintain and whose inner workings are hard to understand or customise. On the other hand, applying deep learning techniques to biomedical data is becoming increasingly popular and more accessible, while delivering promising results. In the specific case of image analysis of immune cells, we want to explore how deep learning could be used to systematically extract metrics of immune cell interactions from microscope images. Specifically, the aim is to assess whether using deep learning in the field of immunology can provide useful information on interaction levels between immune cells under different experimental conditions. 